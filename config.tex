\usepackage[a4paper,width=170mm,top=25mm,bottom=15mm]{geometry}

\usepackage{graphics}
\usepackage{thesis}
\usepackage{latexsym}
\usepackage{amssymb}
\usepackage[dvips]{epsfig}
\graphicspath{{./fig/}}

% Chinese Only Begins
\usepackage{xeCJK}
\usepackage{fontspec}
\setCJKmainfont[BoldFont={FandolHei},ItalicFont={cwTeXKai}]{cwTeXMing}
% Chinese Only Ends


\usepackage{lipsum}
\usepackage[english]{babel}
\usepackage{amsmath, amsfonts, amssymb, amsthm, dsfont, graphicx, tabularx, adjustbox, graphics, bbm, mathrsfs, accents, upgreek, enumitem, mathtools, multirow, nameref, natbib, setspace, titlesec}
\usepackage[dvipsnames]{xcolor}
\usepackage[page,title]{appendix}
\usepackage[font=footnotesize,justification=centering]{caption}
%% todonotes
\usepackage[colorinlistoftodos,prependcaption,textsize=tiny]{todonotes}
\usepackage{xargs}
\newcommandx{\unsure}[2][1=]{\todo[linecolor=red,backgroundcolor=red!25,bordercolor=red,#1]{#2}}
\newcommandx{\change}[2][1=]{\todo[linecolor=blue,backgroundcolor=blue!25,bordercolor=blue,#1]{#2}}
\newcommandx{\info}[2][1=]{\todo[linecolor=OliveGreen,backgroundcolor=OliveGreen!25,bordercolor=OliveGreen,#1]{#2}}
\newcommandx{\improvement}[2][1=]{\todo[linecolor=Plum,backgroundcolor=Plum!25,bordercolor=Plum,#1]{#2}}
\newcommandx{\thiswillnotshow}[2][1=]{\todo[disable,#1]{#2}}

\usepackage{xr}
\usepackage[colorlinks=true, allcolors=blue, urlcolor=black, citecolor=blue]{hyperref}
\usepackage[normalem]{ulem}
\usepackage{subfiles}

% use International Phonetic Alphabet
\usepackage{tipa}
\newcommand{\wt}[1]{\textipa{/#1/}}

% product integral
\usepackage{prodint}

\usepackage{arydshln}
% Another combination of values
\setlength\dashlinedash{1pt}
\setlength\dashlinegap{2pt}

\usepackage{tabularx,booktabs}
\newcolumntype{Y}{>{\centering\arraybackslash}X}

% Use these for theorems, lemmas, proofs, etc.
\theoremstyle{definition} 
\newtheorem{theorem}{Theorem}%[section] %% subsection can be changed to section
\newtheorem{lemma}[theorem]{Lemma}
\newtheorem{proposition}[theorem]{Proposition}
\newtheorem{claim}[theorem]{Claim}
\newtheorem{corollary}[theorem]{Corollary}
\newtheorem{definition}[theorem]{Definition}
\newtheorem{assumption}[theorem]{Assumption}
\newtheorem{example}[theorem]{Example}
\newtheorem{remark}{Remark}
\newtheorem{exercise}[theorem]{Exercise}
\newtheorem{observation}[theorem]{Observation}
\newtheorem*{theorem*}{Theorem}
\newtheorem*{lemma*}{Lemma}
\usepackage[nameinlink]{cleveref}
\crefname{enumi}{}{}
\Crefname{subsubsubappendix}{Appendix}{Appendices}%
\makeatletter
\addto\extrasenglish{\def\itemautorefname{\@gobble}}
\makeatother

%% code from mathabx.sty and mathabx.dcl
\DeclareFontFamily{U}{mathx}{\hyphenchar\font45}
\DeclareFontShape{U}{mathx}{m}{n}{
      <5> <6> <7> <8> <9> <10>
      <10.95> <12> <14.4> <17.28> <20.74> <24.88>
      mathx10
      }{}
\DeclareSymbolFont{mathx}{U}{mathx}{m}{n}
\DeclareFontSubstitution{U}{mathx}{m}{n}
\DeclareMathAccent{\widecheck}{0}{mathx}{"71}
\DeclareMathAccent{\wideparen}{0}{mathx}{"75}


\usepackage{stackengine}
\newcommand\xrowht[2][0]{\addstackgap[.5\dimexpr#2\relax]{\vphantom{#1}}}

%% for cross-referncing between files
% \makeatletter
% \newcommand*{\addFileDependency}[1]{% argument=file name and extension
%   \typeout{(#1)}
%   \@addtofilelist{#1}
%   \IfFileExists{#1}{}{\typeout{No file #1.}}
% }
% \makeatother
% \newcommand*{\myexternaldocument}[1]{%
%     \externaldocument{#1}%
%     \addFileDependency{#1.tex}%
%     \addFileDependency{#1.aux}%
% }
% \myexternaldocument{appendix}

%%% for making comments
% for easy referencing in the manuscript, comment out when finalize
% \usepackage[left]{lineno}
% \usepackage{etoolbox} %% <- for \pretocmd, \apptocmd and \patchcmd
% \linenumbers
% %% Patch AMS math environment:
% \newcommand*\linenomathpatchAMS[1]{%
%   \expandafter\pretocmd\csname #1\endcsname {\linenomathAMS}{}{}%
%   \expandafter\pretocmd\csname #1*\endcsname{\linenomathAMS}{}{}%
%   \expandafter\apptocmd\csname end#1\endcsname {\endlinenomath}{}{}%
%   \expandafter\apptocmd\csname end#1*\endcsname{\endlinenomath}{}{}%
% }
% %% Definition of \linenomathAMS depends on whether the mathlines option is provided
% \expandafter\ifx\linenomath\linenomathWithnumbers
%   \let\linenomathAMS\linenomathWithnumbers
%   %% The following line gets rid of an extra line numbers at the bottom:
%   \patchcmd\linenomathAMS{\advance\postdisplaypenalty\linenopenalty}{}{}{}
% \else
%   \let\linenomathAMS\linenomathNonumbers
% \fi
% \linenomathpatchAMS{gather}
% \linenomathpatchAMS{multline}
% \linenomathpatchAMS{align}
% \linenomathpatchAMS{alignat}
% \linenomathpatchAMS{flalign}



\setcounter{tocdepth}{3}
\setcounter{secnumdepth}{3}

\usepackage[titles]{tocloft}
\cftsetindents{figure}{0em}{3.5em}
\cftsetindents{table}{0em}{3.5em}



\def\biblio{\bibliographystyle{asa}\bibliography{bib}}

\topmargin	0.0cm
\textheight    21cm
%\renewcommand{\arraystretch}{1.2}


%=======================================================